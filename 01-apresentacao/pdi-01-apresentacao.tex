\documentclass[
size=17pt,
paper=smartboard,
mode=present,
display=slidesnotes,
style=paintings,
nopagebreaks,
blackslide,
fleqn]{powerdot}

% styles: sailor, paintings
% wj capsules prettybox
% mode = handout or present


\usepackage{amsmath,graphicx,color,amsfonts}
\usepackage[brazilian]{babel}
\usepackage[utf8]{inputenc}
\newcommand{\palette}{Moitessier}


% palettes:
%    - sailor: Sea, River, Wine, Chocolate, Cocktail 
%    - paintings: Syndics, Skater, GoldenGate, Moitessier, PearlEarring, Lamentation, HolyWood, Europa, MayThird, Charon 

\newcommand{\cursopequeno}{EC01039 CG\underline{PI}}
\newcommand{\cursogrande}{\Large EC01039 -- Computação gráfica e \underline{processamento de imagem}}



\author{Ronaldo de Freitas Zampolo\\FCT-ITEC-UFPA}
\date{2020.2}


\pdsetup{
	lf = {\cursopequeno},
	cf = {\theslide},
   rf = {Apresentação do curso}, palette = {\palette}, randomdots={false}
}

%opening
\title{\cursogrande\\ \vspace{1cm}Apresentação do curso}

\begin{document}
   \maketitle[randomdots={false}]
   \begin{slide}{Agenda}
      \tableofcontents[content=sections]
   \end{slide}

   \section[ slide = true]{Professor e atendimento}
      \begin{slide}[toc=]{Professor e atendimento}
         \begin{itemize}[type=1]
            \item Professor: Ronaldo de Freitas Zampolo 
            \item Afiliação:\\
                  Laboratório de Processamento de Sinais - LaPS\\
                  Faculdade de Engenharia da Computação e Telecomunicações - FCT\\
                  Instituto de Tecnologia - ITEC\\
                  Universidade Federal do Pará - UFPA
            \item Atividades síncronas:\\
                  Segundas, quartas e sextas: 14h50\\
                  %https://conferenciaweb.rnp.br/webconf/ronaldo-de-freitas-zampolo\\
                  \texttt{zampolo@ufpa.br}\\ 
                  \texttt{www.laps.ufpa.br}
         \end{itemize}
      \end{slide}

   \section[ slide = true]{Características do curso}
      \begin{slide}[toc=]{Características do curso}
         \begin{itemize}[type=1]
            \item Carga horária: 30 h%\pause 
%            \item Encontros presenciais: segunda, às 14h50%\pause 
            \item Teoria:
            \begin{itemize}
               \item Pré-requisito desejável: disciplina de PDS
               \item Extensão dos conceitos e ferramentas de PDS
               \item Exercícios teóricos semanais %\pause
            \end{itemize}
            \item Prática:
            \begin{itemize}
               \item Aspecto que receberá a maior ênfase
               \item Implementações semanais 
               \item Linguagem de programação: Python 
            \end{itemize}
         \end{itemize}         
      \end{slide}

   \section[ slide = true]{Tópicos do curso}
   \begin{slide}[toc=]{Tópicos}
      \begin{itemize}[type=1]
         \item Introdução ao processamento de imagem (Cap. 1)
         \item Fundamentos da imagem digital (Cap.2)
         \item Transformações de intensidade e filtragem espacial (Cap. 3)
         \item Filtragem no domínio da frequência (Cap. 4) 
         \item Restauração e reconstrução de imagens (Cap. 5)
         %\item Processamento de imagens coloridas (Cap. 6)
         \item Compressão de imagens (Cap. 8)
         %\item Processamento morfológico de imagens (Cap. 9)
         \item Segmentação de imagens (Cap. 10)
         %\item Análise de imagem
      \end{itemize}
   \end{slide}

   \section[slide=true]{Objetivos, habilidades e competências}
      \begin{slide}[toc=]{Objetivos, habilidades competências}
         \begin{itemize}
            \item Objetivos:
		    \begin{itemize}
			    \item Conhecer metodologias e conceitos básicos sobre PDI/VC
			    \item Implementar sistemas de PDI usando linguagem de alto nível
			    \item Projetar e implementar filtros 2D
			    \item Implementar um sistema simples para compressão (imagem e/ou vídeo) 
		    \end{itemize}
	     \item Habilidades e competências:
		     \begin{itemize}
			     \item Implementação de procedimentos para leitura, exibição e armazenamento de imagens digitais usando linguagens de programação de alto nível
			     \item Implementação de procedimentos de transformação de intensidade de pixel 
			     \item Implementação de filtros simples de suavização e/ou realce no domínio espacial e da frequência 
			     \item Conhecimento das principais metodologias de segmentação de imagens %processamento morfológico e 
			     \item Capacidade para combinar procedimentos simples para atender às necessidades de uma aplicação
			     \item Entendimento sobre os princípios básicos usados em compressão de imagens
	            \end{itemize}
         \end{itemize}
   \end{slide}


   \section[ slide = true]{Bibliografia}
   \begin{slide}[toc=]{Bibliografia}
      \begin{itemize}
       \item Bibliografia básica
       \begin{itemize}
          \item R. C. Gonzalez, R. E. Woods, \emph{Processamento digital de imagens}, 3ª edição, Pearson, 2010.
          \item H. Pedrini, W. R. Schwartz, \emph{Análise de imagens digitais: princípios, algoritmos e aplicações}, 1a edição, Thomson Learning, 2008.
       \end{itemize}
       \item Bibliografia complementar
       \begin{itemize}
          \item T. Acharya, A. K. Ray \emph{Image processing - principles and applications}, John-Wiley,  2005.
          \item W. K. Pratt, \emph{Digital image processing}, 4ª edição, John-Wiley,  2007.
          \item J. W. Woods, \emph{Multidimensional signal, image and video processing and coding}, Academic Press,  2006.
       \end{itemize}
      \end{itemize}
   \end{slide}
  
  \section[ slide = true]{Metodologia, ferramentas e avaliação}
      \begin{slide}[toc=]{Metodologia, ferramentas e avaliação}
         \begin{itemize}
	    \item Metodologia utilizada: aula invertida  
		    \begin{itemize}
			    \item Atividades em sala de aula: resolução de exercícios, aprofundamento do conteúdo
			    %\item Chat no SIGAA (atividade síncrona): plantão de dúvidas
			    \item Maior parte do estudo será feita de maneira guiada, mas assíncrona: leituras, vídeos, atividades diversas
		    \end{itemize}
	    \item Ferramentas usadas:
		    \begin{itemize}
			    %\item Atividades síncronas: Google Meet 
			    \item SIGAA: repositório e sistema de entrega
			    \item Google Colaboratory (Python): simulações e programação 
		    \end{itemize}

            \item Avaliação continuada
            \begin{itemize}
               \item Trabalhos 
               \item Tarefas 
	       \item Listas de exercício
	    \end{itemize}
	 \end{itemize}
      \end{slide}

\end{document}
